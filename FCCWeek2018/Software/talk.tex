%%%%%%%%%%%%%%%%%%%%%%%%%%%%%%%%%%%%%%%%%
% Beamer Presentation
% LaTeX Template
% Version 1.0 (10/11/12)
%
% This template has been downloaded from:
% http://www.LaTeXTemplates.com
%
% License:
% CC BY-NC-SA 3.0 (http://creativecommons.org/licenses/by-nc-sa/3.0/)
%
%%%%%%%%%%%%%%%%%%%%%%%%%%%%%%%%%%%%%%%%%

%----------------------------------------------------------------------------------------
%	PACKAGES AND THEMES
%----------------------------------------------------------------------------------------

\documentclass[hyperref={colorlinks=true,pdfpagelabels=false,linkcolor=black}, xcolor=dvipsnames,10pt]{beamer}
\hypersetup{pdftex=true, colorlinks=true, breaklinks=true, linkcolor=black, menucolor=blue, pagecolor=blue, urlcolor=cyan}


\usepackage{graphicx} % Allows including images
\usepackage{booktabs} % Allows the use of \toprule, \midrule and \bottomrule in tables
\usepackage{xcolor, color}% ,colortbl}
\usepackage{tikz} % nalipour
\usetikzlibrary{tikzmark} % nalipour
\usepackage{hyperref}
\usetikzlibrary{shapes,arrows, decorations.pathreplacing} %nilou
\usetikzlibrary{decorations.markings} %nilou
\usetikzlibrary{trees}%nilou
\usetikzlibrary{decorations.pathmorphing}%nilou 
\usepackage{siunitx}%nilou
\usepackage{framed}%nilou
\usetikzlibrary{backgrounds} %nilou
\usepackage[framemethod=tikz]{mdframed}%nilou
\usepackage{adjustbox} %nilou for adjusting the table
\usepackage{amssymb} %nalipour for checkmark
\usepackage{libertine} % For the font

%\usepackage{../../../CLICdp_definitions} % nalipour
\usepackage{xspace}
\usepackage{upgreek} % nalipour
\usepackage{amsmath, mathtools} % nalipour
\renewcommand{\thefootnote}{\fnsymbol{footnote}} % nalipour: symbols
                                % for the footnote
\usepackage{verbatim} % nalipour
\usepackage{fixltx2e}
\usepackage{adjustbox}%nalipour
\usepackage{pifont} %nalipour
\usepackage{smartdiagram}
\usesmartdiagramlibrary{additions}
\usepackage{siunitx}

\usetikzlibrary{angles,quotes}
\usetikzlibrary{positioning}

%%%%%%%%%%%%%%%%%%%%%%%%%%%%%

%----------------------------------------------------------------------------------------
%	TITLE PAGE
%----------------------------------------------------------------------------------------
\title[]{Implementation of the IDEA drift chamber within FCCSW}
\author[Niloufar Alipour Tehrani]{Niloufar Alipour Tehrani
  \vspace{0.3cm} }

\institute[CERN]{} 
\date[14 March 2018]{FCC-ee MDI study group meeting\\ \vspace{0.3cm}
  \scriptsize{CERN \\
14 March 2018}}


\setbeamertemplate{navigation symbols}{}
\setbeamertemplate{footline}[frame number]

\usetheme{default}%CambridgeUS}%Boadilla}%Pittsburgh}
\usecolortheme{default}

\setlength{\leftmargini}{2pt} % nalipour: left margin indentation
\renewcommand{\inserttotalframenumber}{\ref{lastframe}} 
\begin{document}
\renewcommand{\inserttotalframenumber}{\pageref{lastslide}}

\tikzset{cross/.style={cross out, draw=black, minimum
    size=2*(#1-\pgflinewidth), inner sep=0pt, outer sep=0pt}, %default
  radius will be 1pt.  cross/.default={1pt}}


%%%%%%%%%%%%%%%%%%%%%%%%%%%%%
%         SLIDE             %
%%%%%%%%%%%%%%%%%%%%%%%%%%%%%
\begin{frame}[plain]
  
  \titlepage
  \begin{columns}
    \column{0.25\textwidth}
    \centering
    \includegraphics[width=\textwidth]{../logos/FCC-logo}
    \column{0.5\textwidth}
    \column{0.25\textwidth}
    \centering
    \includegraphics[width=0.6\textwidth]{../logos/logo_cern.pdf}
  \end{columns}
\end{frame}




%%%%%%%%%%%%%%%%%%%%%%%%%%%%%
%         SLIDE             %
%%%%%%%%%%%%%%%%%%%%%%%%%%%%%
\begin{frame}
	\frametitle{Introduction}
	
	\begin{enumerate}
	\item FCC experiment and FCCSW
	\item IDEA detector concept
	\item Drift chamber
	\item Geometry implementation
	\item Segmentation and validation
	\item Simulation
	\item Where to find and how to simulate the drift chamber
	\item Material budget scan
	\item Background studies
	\end{enumerate}
	

%		\centering
%    		\includegraphics[width=\textwidth]{figures/cellid2359297}


\end{frame}


%%%%%%%%%%%%%%%%%%%%%%%%%%%%%
%         SLIDE             %
%%%%%%%%%%%%%%%%%%%%%%%%%%%%%
\label{lastslide}
\begin{frame}
  \frametitle{Conclusions}

\end{frame}





%----------------------------------------------------------------------------------------
%	End of Document
%----------------------------------------------------------------------------------------
\end{document} 
